% Chapter Template

\chapter{Introduction} % Main chapter title

% \label{Intro} % Change X to a consecutive number; for referencing this chapter elsewhere, use \ref{ChapterX}

\lhead{Chapter 1. \emph{Introduction}} % Change X to a consecutive number; this is for the header on each page - perhaps a shortened title

%----------------------------------------------------------------------------------------
%	SECTION 1
%----------------------------------------------------------------------------------------
\section{Problem Statement}

What problem is being solved with the software HMMER3?

Why is the problem important?

Why is an optimized implementation needed?

What is a protein sequence profile HMM, etc

What are the common use cases



HMMER \citep{HMMER} is a free and commonly used software package for sequence analysis written by Sean Eddy. 
It is an open source implementation of HMM algorithms for use with protein databases.

One of its more widely used applications, \emph{hmmsearch} is to identify homologous protein. It does this with Viterbi algorithm described in subsection \ref{ViterbiSub} by comparing a profile HMM to each protein sequence in a large database, evaluating the path that has the maximum probability of the HMM generating the sequence.
This search requires a computationally intensive procedure.

There has been a great deal of work on optimizing HMMER for both CPUs and GPUs. 

JackHMMer \citep{Wun} uses the Intel IXP 2850 network processor to accelerate Viterbi algorithm. The processor is used as a single-chip cluster with the XScale CPU functioning as the head node. Like a typical cluster, the XScale CPU is responsible for distributing jobs to the individual microengines. 

MPI-HMMER \citep{Walters2006} is a wellknown and commonly used MPI implementation. In their studies, a single master node is used to assign multiple database blocks to worker nodes for computing in parallel. And it is responsible for collecting the results.

HMMER3 \citep{HMMER3} is the most significant acceleration of hmmsearch. The main performance gain is due to a heuristic algorithm called MSV filter, for Multiple (local, ungapped) Segment Viterbi, as described in Section\ref{MSVsub}. MSV is implemented in SIMD vector parallelization instructions and is about 100-fold faster than HMMER2.

GPUs have been shown to provide very attractive compute resources in addition to CPUs, because of particular manycore parallel computation in GPUs.

\citep{GPUHMM}, \citep{Ganesan}, \citep{Du} and \citep{Quirem} parallelized Viterbi algorithm on CUDA-enabled GPUs.

[Ahmed et al., 2012]\citep{Ahmed} used Intel VTune Analyzer \citep{Intel} to investigate performance hotspot functions in HMMER3. Based on hotspot analysis, they studied CUDA acceleration for three individual algorithm: Forward, Backward and Viterbi algorithms.

As shown in Figure \ref{fig:hmmsearch}, the MSV and Viterbi algorithms are implemented in the so-called ``acceleration pipeline" at the core of the HMMER3 software package \citep{HMMER3}. And the MSV algorithm is the first filter of ``acceleration pipeline" and is the key hotspot of the whole process. Therefore, this thesis concentrate on porting the MSV onto CUDA-enabled GPU to accelerate hmmsearch application.

\section{Research Contributions}
The contribution of this thesis can be classified as follows:
\begin{itemize}
 \item Analyze the core application \emph{hmmsearch} in HMMER3 and find the key hotspot MSV filter for accelerating hmmsearch.
 \item Implement the protein sequence search tool \emph{cudaHmmsearch} on CUDA-enabled GPU. Demonstrate many optimization approaches to accelerate cudaHmmsearch.
 \item Discuss and analyze the advantages and limitations of GPU hardware for CUDA parallel programming.
\end{itemize}

\section{Organization of thesis}
The rest of this thesis is organized as follows:

Chapter \ref{Background} introduces the background necessary for understanding the work in this thesis.

Then Chapter \ref{CUDAHMMER3} presents the details of our \emph{cudaHmmsearch} implementation and optimization approaches. And 6 steps are summarized for better performance of CUDA programming at the end of this Chapter.

Comprehensive benchmarks were performed and analyzed in Chapter \ref{Results}. 

The conclusion of Chapter \ref{Conclusions} summarizes our contributions, points out its limitations, and makes suggestions for future work.

\section{Typographical Conventions}
The following font conventions are used in this thesis:
\begin{itemize}
 \item {\fontfamily{phv}\fontseries{m}\selectfont Adobe Helvetica font}\\
 Used for code examples.
 \item {\fontfamily{phv}\fontseries{m}\selectfont \textsl{Adobe Helvetica slanted font}}\\
 Used for comments of code.
 \item {\fontfamily{pag}\selectfont Adobe AvantGarde font}\\
 Used for captions of table and figure.
\end{itemize}
