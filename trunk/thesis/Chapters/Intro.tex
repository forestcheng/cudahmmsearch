% Chapter Template

\chapter{Introduction} % Main chapter title

% \label{Intro} % Change X to a consecutive number; for referencing this chapter elsewhere, use \ref{ChapterX}

\lhead{Chapter 1. \emph{Introduction}} % Change X to a consecutive number; this is for the header on each page - perhaps a shortened title

%----------------------------------------------------------------------------------------
%	SECTION 1
%----------------------------------------------------------------------------------------
\section{Problem Statement}
What's it + why important

History of problem + state of art

Summary of work + contribution

Discuss common use of HMMER

HMMER \citep{HMMER} is a free and commonly used software package for sequence analysis written by Sean Eddy. 
It is an open source implementation of HMM algorithms for use with protein databases.

One of its more widely used applications, \emph{hmmsearch} is to identify homologous protein. It does this with Viterbi algorithm by comparing a profile HMM to each protein sequence in a large database, evaluating the path that has the maximum probability of the HMM generating the sequence.
This search requires a computationally intensive procedure.

There has been a great deal of work on optimizing HMMER for both CPUs and GPUs since the stable HMMER1 was developed in 2005. 
JackHMMer \citep{Wun} uses the Intel IXP 2850 as a singlechip cluster with the XScale CPU functioning as the head node. Like a typical cluster, the XScale CPU is responsible for distributing jobs to the individual microengines. 


\citep{Walters2006} is a wellknown and commonly used MPI implementation. In their studies, a single master node is used to assign multiple database blocks to worker nodes for computing in parallel. And it is responsible for collecting the results.


HMMER3 \citep{HMMER3} is the most significant acceleration of hmmsearch. The main performance gain is due to a heuristic algorithm called MSV filter, for Multiple (local, ungapped) Segment Viterbi. MSV is implemented in SIMD vector parallelization instructions and is about 100-fold faster than HMMER2.

GPUs have been shown to provide very attractive compute resources in addition to CPUs, because of particular manycore parallel computing in GPUs.
\citep{GPUHMM}, \citep{Ganesan}, \citep{Du} and \citep{Quirem}
[Walters et al., 2009], [Ganesan et al., 2010], [Du et al., 2010] and [Quirem et al., 2011]
parallelized Viterbi algorithm on CUDA-enabled GPUs.
[Ahmed et al., 2012]\citep{Ahmed} used Intel VTune Analyzer \citep{Intel} to investigate performance hotspot functions in HMMER3. Based on hotspot analysis, they studied CUDA acceleration for three individual algorithm: Forward, Backward and Viterbi algorithms.

of hmmsearch in HMMER3 is shown in Figure\ref{fig:hmmsearch}. The MSV and Viterbi algorithms described in subsection \ref{ViterbiSub} and \ref{MSVsub}

As shown in Figure\ref{fig:hmmsearch}, the MSV and Viterbi algorithms are implemented in the so-called ``acceleration pipeline" at the core of the HMMER3 software package \citep{HMMER3}. And the MSV algorithm is the first filter of "acceleration pipeline" and is the key hotspot of the whole process. Therefore, our research concentrate on porting the MSV onto CUDA-enabled GPU to accelerate hmmsearch application.

HMMER \citep{HMMER} is a set of applications that create a profile Hidden Markov Model (HMM) of a sequence family which can be utilized as a query against a sequence database to identify (and/or align) additional homologs of the sequence family\citep{Seq}. HMMER was developed by Sean Eddy at Washington University and has become one of the most widely used software tools for sequence homology.

HMMer [Eddy 2003a]  One of the more widely used algorithms,
hmmsearch, works as follows: a user provides an
HMM modeling a desired protein family and hmmsearch
processes each protein sequence in a large database, evaluating
the probability that the most likely path through the
query HMM could generate that database protein sequence.
This search requires a computationally intensive procedure,
known as the Viterbi [1967; 1973] algorithm. The search
could take hours or even days depending on the size of the
database, query model, and the processor used.

hmmsearch is widely used in the biology
research community.



\section{Research Contributions}

\section{Organization of thesis}

\subsection*{Typographical Conventions}
The following font conventions are used in this thesis:
\begin{itemize}
 \item {\fontfamily{phv}\fontseries{m}\selectfont Adobe Helvetica font}\\
 Used for code examples.
 \item {\fontfamily{phv}\fontseries{m}\selectfont \textsl{Adobe Helvetica slanted font}}\\
 Used for comments of code.
 \item {\fontfamily{pag}\selectfont Adobe AvantGarde font}\\
 Used for captions of table and figure.
\end{itemize}


%-----------------------------------
%	SUBSECTION 1
%-----------------------------------
\subsection{Subsection 1}


%-----------------------------------
%	SUBSECTION 2
%-----------------------------------

\subsection{Subsection 2}

%----------------------------------------------------------------------------------------
%	SECTION 2
%----------------------------------------------------------------------------------------

\section{Main Section 2}

