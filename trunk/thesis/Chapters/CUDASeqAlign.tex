% Chapter 3

\chapter{CUDA accelerated sequence alignment} % Main chapter title

\label{CUDASeqAlign} % For referencing the chapter elsewhere, use \ref{Chapter1} 

\lhead{Chapter 3. \emph{Dynamic programming in Bioinformatics}} % This is for the header on each page - perhaps a shortened title

%----------------------------------------------------------------------------------------

\section{Overview of CUDA-enabled GPU hardware}


OpenCL™ (Open Computing Language) is the first truly open and royalty-free programming standard for general-purpose computations on heterogeneous systems \citep{OpenCL}. OpenCL™ is maintained by the non-profit technology consortium Khronos Group. It has been adopted by many corporations, including Nvidia, Apple, Intel, Qualcomm, AMD, Altera, Samsung, Vivante and ARM Holdings.

Microsoft Compute Shader\citep{Shader}, also known as  DirectCompute technology, is used for GPU computing and supported on NVIDIA’s DX10 and DX11 class GPUs under Windows VISTA and later versions of Windows\citep{DirectCompute}. Compute Shader is designed and implemented with HLSL(High Level Shading Language)\citep{HLSL}, providing memory sharing and thread synchronization features to allow more effective parallel programming methods.

%----------------------------------------------------------------------------------------

\section{Overview of CUDA programming model}



%----------------------------------------------------------------------------------------

\section{CUDA accelerated Smith-Waterman}

%----------------------------------------------------------------------------------------

\section{CUDA accelerated HMMER}

%----------------------------------------------------------------------------------------
