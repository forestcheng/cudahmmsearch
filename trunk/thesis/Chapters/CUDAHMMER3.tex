% Chapter 4

\chapter{A CUDA accelerated HMMER3 protein sequence search tool} % Main chapter title

\label{CUDAHMMER3} % For referencing the chapter elsewhere, use \ref{Chapter1} 

\lhead{Chapter \ref{CUDAHMMER3}. \emph{A CUDA accelerated HMMER3 protein sequence search tool}} % This is for the header on each page - perhaps a shortened title

%----------------------------------------------------------------------------------------

\section{Requirements and design decisions}

The following are the requirements for a CUDA accelerated HMMER3 protein sequence search tool:

\begin{itemize}
\item A HMMER3 protein sequence search tool, named \emph{cudaHmmsearch}, will be implemented to run on CUDA-enabled GPU and several optimization will be taken to accelerate the computation. The cudaHmmsearch will be tested and compared with other CPU and GPU implementations.
\item The cudaHmmsearch will be based on the HMMER3 algorithm so that the result will be the same as hmmsearch of HMMER3.
\item The cudaHmmsearch will be completely usable under various GPU devices and sequence database sizes. This means that cudaHmmsearch is not just for research purpose or just a proof of concept.
\end{itemize}

\subsubsection*{Implementation toolkit and language}

NVIDIA CUDA \citep{CUDAzone} was chosen as the toolkit to be used in the implementation phase. Since its introduction in 2006, CUDA has been widely deployed through thousands of applications and published research papers, and supported by an installed base of over 500 million CUDA-enabled GPUs in notebooks, workstations, compute clusters and supercomputers \citep{CUDAwhat}. As of writing, CUDA is the most mature and popular GPU programming toolkit. 

HMMER3 is implemented in the C programming language. CUDA provides a comprehensive development environment for C and C++ developers. However, some advanced features of CUDA, such as texture, are only supported in C++ template programming. So C++ has to be used, and some compatibility problems when compiling and programming between C and C++ have also to be dealt with accordingly.

\subsubsection*{Implementation methods}
\label{impl}

The following two approaches have been explored for parallelizing the protein sequence database search using CUDA. A target sequence in the database is processed as one task.

\begin{itemize}
 \item \textbf{Task-based parallelism} Each task is assigned to exactly one thread, and \emph{blockDim} tasks are performed in parallel by different threads in a thread block.
 \item \textbf{Data-based parallelism} Each task is assigned to one or many thread block(s) and all threads in the thread block(s) cooperate to perform the task in parallel.
\end{itemize}

Task-based parallelism has some advantages over data-based parallelism. On one hand, it removes the need for inter-thread communications or, even worse, inter-multiprocessor communications. As described before, the thread or processing elements of one CUDA multiprocessor can communicate by using shared memory, while slow global memory must be used to transfer data between multiprocessors. At the same time, data-based parallelism also needs to take time on synchronizing and cooperating among threads. On the other hand, task-based parallelism is processing one sequence on each thread which results in a kernel where each processing element is doing the exact same thing independently. This also simplifies implementation and testing. Although task-based parallelism needs more device memory than data-based parallelism, it can achieve better performance \citep{SW++}. Thus, the approach of task-based parallelism is taken and efforts are focused on optimizing CUDA kernel execution.

Since data-based parallelism occupies significantly less device memory, \citep{SW++} uses it to support the longest query/subject sequences. However, a different strategy here is applied to work around this problem and is discussed in detail in subsection \ref{workload} on workload distribution.

%----------------------------------------------------------------------------------------

\section{A straightforward implementation}
\label{Aimpl}
This section describes a straightforward, mostly un-optimized implementation of the protein database search tool. First, a simple serial CPU implementation of hmmsearch is presented, with no GPU specific traits. Next, the MSV filter is ported to the GPU. This implementation is then optimized in the next section.

\subsection{CPU serial version of hmmsearch}

The CPU serial version of hmmsearch in HMMER3 is shown in Figure \ref{fig:hmmsearch}. The MSV and Viterbi algorithms described in subsection \ref{ViterbiSub} and \ref{MSVsub} are implemented in the so-called “acceleration pipeline” at the core of the HMMER3 software package \citep{HMMER3}. One call to the acceleration pipeline is executed for the comparison of each query model and target sequence.

\begin{figure}[!htb]
 \centering
 \includegraphics[totalheight=0.5\textheight]{Figures/hmmsearch.png}
 \caption{\fontfamily{pag}\selectfont The CPU serial version of hmmsearch}
 \label{fig:hmmsearch}
\end{figure}

\label{hmmsearch}
After each filter step, the pipeline either accepts or rejects the entire comparison, based on the P-value of the score calculated in each filter. For example, as can be seen in Figure \ref{fig:hmmsearch}, by default a target sequence can pass the MSV filter if its comparison gets a P-value of less than 0.02. In practice about 2\% of the top-scoring target sequences are expected to pass the filter. So, much fewer target sequences can pass one filter and hence need further computing. In consequence, the comparison is accelerated. Thus, the first MSV filter is typically the run time bottleneck for hmmsearch. Therefore, the key to parallelizing hmmsearch tool is to offload the MSV filter function to multiple computing elements on the GPU, while ensuring that the code shown in Figure \ref{MSV-SIMD} is as efficient as possible.

\subsection{GPU implementation of MSV filter}

A basic flow of the GPU implementation for the MSV filter is shown in Figure \ref{fig:gpuMSV}. The code is split up into two parts, with the left \emph{host} part running on the CPU and the right \emph{device} part running on the GPU. There is some redundancy as data needed by the GPU will be copied between the memories in the host and the device.

\begin{figure}[!htb]
 \centering
 \includegraphics[totalheight=0.6\textheight]{Figures/gpuMSV.png}
 \caption{\fontfamily{pag}\selectfont The GPU porting of MSV filter}
 \label{fig:gpuMSV}
\end{figure}

The CPU code mainly concerns allocating data structures on the GPU, loading data, copying data to the GPU, launching the GPU kernel and copying back the results for further steps.

The GPU kernel code corresponds to the MSV filter Algorithm \ref{MSV-SIMD}. First, the thread's current database sequence is set to the thread id. Hence each thread begins processing a different neighbouring sequence. This thread id is a unique numeric identifier for each thread and the id numbers of threads in a warp are consecutive. Next, the location where each thread can store and compute its dp matrix is determined in the global memory. This is calculated also using the thread id for each thread. When processing the sequence, successive threads access the successive addresses in the global memory for the sequence data and dp matrix, i.e. using a coalesced access pattern. Execution on the GPU kernel is halted when every thread finishes its sequence.

%----------------------------------------------------------------------------------------

\section{Optimizing the implementation}

Although a fully functioning GPU MSV filter has been presented, its simple implementation is quite slow: more than 227 seconds to search the test database Swiss-Prot with 540,958 query sequences, as shown in Table \ref{tab.opt}.

This section discusses the optimization steps taken to eventually reach a benchmark database search time of 1.65 seconds: an almost 137 times speedup.

\subsection{Global Memory Accesses}
\label{global}

The global memory is used to store most of data on the GPU. A primary concern in the optimization is to improve the efficiency of accessing global memory. One way is to reduce the frequency of access. Another way is coalescing access.

\subsubsection*{Access frequency}
\label{Afre}
The elements of the \emph{dp} matrix and the query profile matrix are 8-bit values.  The \emph{uint4} and \emph{ulong2} (see the code below) are 128-bit CUDA built-in vector types. So the access frequency would be decreased 16 times by using \emph{uint4} or \emph{ulong2} to fetch the 8-bit values residing in global memory, compared with using 8-bit \emph{char} type.

\begin{quote}
\fontfamily{phv}\fontseries{m}\selectfont
struct \_\_device\_builtin\_\_ uint4\\
\{\\
   unsigned int x, y, z, w;\\
\}\\
struct \_\_device\_builtin\_\_ ulong2\\
\{\\
    unsigned long int x, y;\\
\};\\
\end{quote}
% \newcommand\codeHighlight[1]{\textcolor[rgb]{0,0,1}{\textbf{#1}}}
% \begin{Verbatim}[commandchars=\\\{\}]
% \codeHighlight{struct}  __device_builtin__ uint4
% \{
%     unsigned int x, y, z, w;
% \}
% \codeHighlight{struct} __device_builtin__ ulong2
% \{
%     unsigned long int x, y;
% \};
% \end{Verbatim}

This approach is very effective and gained a huge speed boost of almost 8 times in total.

\subsubsection*{Coalescing access}
\label{coal}
Coalescing access is the single most important performance consideration in programming for CUDA-enabled GPU architectures. Coalescing is a technique applied to combine several small and non-contiguous access of global memory, into a single large and more efficient contiguous memory access. A prerequisite for coalescing is that the words accessed by all threads in a warp must lie in the same segment. As can be seen in Figure \ref{fig:coalescing}, the memory spaces referred to by the same variable names (not referring to the same addresses) for all threads in a warp have to be allocated in the form of an array to keep them contiguous in address space.

\begin{figure}[!htb]
	\centering
	\includegraphics[totalheight=0.115\textheight]{Figures/coalesce.png}
	\caption{\fontfamily{pag}\selectfont \textbf{Coalescing Global Memory Accesses\citep{Waters}}. A prerequisite for coalescing access global memory: the addresses of global memory being accessed by the threads in a warp must be contiguous and increasing (i.e., offset by the thread ID). }
	\label{fig:coalescing}
\end{figure}

For coalescing access, the target sequences are arranged in a matrix like an upside-down bookcase shown in Figure \ref{fig:dbalign}, where all residues of a sequence are restricted to be stored in the same column from top to bottom. And all sequences are arranged in decreasing length order from left to right in the array, which is explained in Section \ref{dbsort}. 

\begin{figure}[!htb]
	\centering
	\includegraphics[totalheight=0.2\textheight]{Figures/dbalign.png}
	\caption{\fontfamily{pag}\selectfont \textbf{Alignment of target sequences} For coalescing access, all residues of a target sequence are stored in the same column of the data matrix from top to bottom. Each thread is in charge of processing a sequence in a column.}
	\label{fig:dbalign}
\end{figure}

Figure \ref{fig:dp} presents the similar global memory allocation pattern of \emph{dp} matrix for \emph{M} processing target sequences. Each thread processes independent \emph{dp} array with the same length \emph{Q}. A memory slot is allocated to a thread and is indexed top-to-bottom, and the access to \emph{dp} arrays is coalesced by using the same index for all threads in a warp.

\begin{figure}[!htb]
	\centering
	\includegraphics[totalheight=0.2\textheight]{Figures/dp.png}
	\caption{\fontfamily{pag}\selectfont \textbf{The allocation pattern of dp matrix in global memory}. Each thread is in charge of processing a sequence in a column. The sequences are indexed from $Seq0$ to $Seq_{M-1}$. $M$ is the number of target sequences. The dp matrix has $Q$ rows which corresponds to the profile length.}
	\label{fig:dp}
\end{figure}

An alignment requirement is needed to fulfill for fully coalescing, which means any access to data residing in global memory is compiled to a single global memory instruction. The alignment requirement is automatically fulfilled for the built-in types like \emph{uint4} \citep{CUDA-C}.

The move to vertical alignment of \emph{dp} matrix resulted in an improvement of about 44\%.

\subsubsection*{Note on coding global memory coalescing access}
At the beginning, since \emph{uint4} is 16-bytes data block, the traditional C/C++ memory block copy function \emph{memcpy}() was used to copy data between global memory and register memory, as shown in the following code. The \emph{dp} is the pointer to the address of global memory. The \emph{mpv} and \emph{sv} are \emph{uint4} data type residing in register memory.

\begin{quote}
\fontfamily{phv}\fontseries{m}\selectfont
 memcpy(\&mpv, dp, sizeof(uint4));\\
 memcpy(dp, \&sv, sizeof(uint4));
\end{quote}

However, in practice during CUDA kernel execution, the above \emph{memcpy} involves $16 = sizeof(uint4)$ reads/writes from/to global memory respectively, not one read/write. Switching to the following direct assignment instruction will be one read/write and fully coalesce access global memory, with 81\% improvement over the above \emph{memcpy}().

\begin{quote}
\fontfamily{phv}\fontseries{m}\selectfont
 mpv = *(dp);\\
 *(dp) = sv;
\end{quote}

\subsection{Texture memory}
\label{tex}

The read-only texture memory space is a cached window into global memory that offers much lower latency and does not require coalescing for best performance. Therefore, a texture fetch costs one device memory read only on a cache miss; otherwise, it just costs one read from the texture cache. The texture cache is optimized for 2D spatial locality, so threads of the same warp that read texture addresses that are close together will achieve best performance \citep{CUDA-C}.

Texture memory is well suited to random access. CUDA has optimized the operation fetching 4 values (RGB colors and alpha component, a typical graphics usage) at a time in texture memory. This mechanism is applied to fetch 4 read-only values from the query profile matrix \emph{texOMrbv} with the \emph{uint4} built-in type. Since the data of target sequences is read-only, it can also use texture memory for better performance.

Switching to texture memory for the query profile texOMrbv resulted in about 22\% performance improvement.

\subsubsection*{Restrictions using texture memory}

Texture memory is designed for the GPU graphics processing and therefore is less flexible than the CUDA standard types. It must be declared at compile time as a fixed type, for example \emph{uint4} for the query profile in our case:

\begin{quote}
\fontfamily{phv}\fontseries{mc}\selectfont
 texture$<$uint4, cudaTextureType2D, cudaReadModeElementType$>$ \emph{texOMrbv};
\end{quote}

How the values are interpreted is specified at run time. Texture memory is read-only to the CUDA kernel and must be explicitly accessed via a special texture API (e.g. tex2D(), tex1Dfetch(), etc) and arrays must be bound to textures.

\begin{quote}
\fontfamily{phv}\fontseries{m}\selectfont
 uint4 rsc4 = tex2D(texOMrbv, x, y);
\end{quote}

However, on the CUDA next-generation architecture Kepler, the texture cache gets a special compute path, removing the complexity associated with programming it \citep{Kepler}.

\subsection{SIMD Video Instructions}
\label{video}

Like Intel SSE2 described in subsection \ref{SSE2}, CUDA also provides the scalar SIMD (Single Instruction, Multiple Data) video instructions. These are available on devices of compute capability 3.0. The SIMD video instructions enable efficient operations on pairs of 16-bit values and quads of 8-bit values needed for video processing.

The SIMD video instructions can be included in CUDA programs by way of the assembler, \emph{asm}(), statement.

The basic syntax of an \emph{asm}() statement is:

\begin{quote}
\fontfamily{phv}\fontseries{m}\selectfont
 asm(``template-string" : ``constraint"(output) : ``constraint"(input));
\end{quote}

The following three instructions are used in the implementation. Every instruction operates on quads of 8-bit signed values. The source operands (``op1"� and ``op2"�) and destination operand (``rv"�) are all unsigned 32-bit registers (``u32"�), which is different from 128-bit CPU registers in SSE2. For additions and subtractions, saturation instructions (``sat"�) have been used to clamp the values to their appropriate unsigned ranges.

\begin{quote}
\fontfamily{phv}\fontseries{mc}\selectfont
 \textsl{/* rv[z] = op1[z] + op2[z] (z = 0,1,2,3) */}\\
 asm(``vadd4.u32.u32.u32.sat \%0, \%1, \%2, \%3;" : ``=r"(rv) : ``r"(op1), ``r"(op2), ``r"(0));\\
 \textsl{/* rv = op1 + op2 */}\\
 asm(``vsub4.u32.u32.u32.sat \%0, \%1, \%2, \%3;" : ``=r"(rv) : ``r"(op1), ``r"(op2), ``r"(0));\\
 \textsl{/* rv = max(op1,op2) */}\\
 asm(``vmax4.u32.u32.u32 \%0, \%1, \%2, \%3;" : ``=r"(rv) : ``r"(op1), ``r"(op2), ``r"(0));
% \slshape A narrow slanted f\'ee.\\
\end{quote}

Switching to the SIMD video instructions also achieved a large speedup of nearly 2 times.

These results, as well as the acceleration of HMMER3, show that the parallel vector instructions can greatly improve computing for array or matrix operations. We can also see the limitation of GPU SIMD computing compared to CPU SIMD computing: the GPU can only support 32-bit register operations, which are 4 times less in data bandwidth than that CPU SSE2 128-bit register operations.

\subsection{Virtualized SIMD vector programming model}

Vector programming model plays an important role in operations of array or matrix. On one hand, it can reduce significantly the frequency of memory access. On the other hand, it can utilize the SIMD vector instructions for parallel computing on the CPU or the GPU.

Inspired by the fact that CUDA has optimized the operation fetching a four component RGBA colour in texture memory, the target sequence is re-organized using a packed data format, where four consecutive residues of each sequence are packed together and represented using the \emph{uchar4} vector data type, instead of the \emph{char} scalar data type, as can be seen in Figure \ref{fig:simdvector}(\textit{a}). In this way, four residues are loaded using only one texture fetch, thus significantly improving texture memory throughput. 

\begin{figure}[!htb]
	\centering
	\includegraphics[totalheight=0.2\textheight]{Figures/simdvector.png}
	\caption{\fontfamily{pag}\selectfont \textbf{SIMD vector alignment pattern}: (\textit{a}) For a target sequence, four consecutive residues which are indexed from 0 are packed together and represented using the uchar4 vector data type. (\textit{b}) For a dp array, 16 consecutive bytes which are indexed from 0 are packed together and represented using the uint4 vector data type. A dp array has $Q$ elements which corresponds to the profile length.}
	\label{fig:simdvector}
\end{figure}

Similarly, the dp array and the query profile also use the virtualized SIMD vector allocation pattern, as can be seen in Figure \ref{fig:simdvector}(\textit{b}). The reason why the target sequence only use 4-byte uchar4, not 16-byte uint4 as the dp array is because it will demand much more register memory. And the dp array has been packed in this pattern in HMMER3.

\subsection{Pinned (non-pageable) Memory}
\label{pin}

It is necessary to transfer data to the GPU over the PCI-E data bus. Compared to the access to CPU host memory, this bus is very slow. Pinned memory is memory that cannot be paged (swapped) out to disk by the virtual memory management of the OS. In fact, PCI-E transfer can only be done using pinned memory, and if the application does not allocate pinned memory, the CUDA driver does this in the background for us. Unfortunately, this results in a needless copy operation from the regular (paged) memory to or from pinned memory. We can of course eliminate this by allocating pinned memory ourselves.

In the application, we simply replace \emph{malloc/free} when allocating/freeing memory in the host application with \emph{cudaHostAlloc/cudaFreeHost}.

\begin{quote}
\fontfamily{phv}\fontseries{m}\selectfont
 cudaHostAlloc (void** host\_pointer, size\_t size, unsigned int flags)
\end{quote}

\subsection{Asynchronous memory copy and Streams}
\label{asyn}

\subsubsection*{Asynchronous memory copy}
By default, any memory copy involving host memory is synchronous: the function does not return until after the operation has been completed. This is because the hardware cannot directly access host memory unless it has been page-locked or pinned and mapped for the GPU. An asynchronous memory copy for pageable memory could be implemented by spawning another CPU thread, but so far, CUDA has chosen to avoid that additional complexity.

Even when operating on pinned memory, such as memory allocated with \emph{cudaMallocHost}(), synchronous memory copy must wait until the operation is finished because the application may rely on that behavior. When pinned memory is specified to a synchronous memory copy routine, the driver does take advantage by having the hardware use DMA, which is generally faster \citep{CUDAHand}.

When possible, synchronous memory copy should be avoided for performance reasons. Keeping all operations asynchronous improves performance by enabling the CPU and GPU to run concurrently. Asynchronous memory copy functions have the suffix \emph{Async}(). For example, the CUDA runtime function for asynchronous host to device memory copy is \emph{cudaMemcpyAsync}().

Asynchronous memory copy works well only where either the input or output of the GPU workload is small and the total transfer time is less than the kernel execution time. By this means we have the opportunity to hide the input transfer time and only suffer the output transfer time.

\subsubsection*{Multiple streams}
A CUDA stream represents a queue of GPU operations that get executed in a specific order. We can add operations such as kernel launches, memory copies, and event starts and stops into a stream. The order in which operations are added to the stream specifies the order in which they will be executed. CUDA streams enable CPU/GPU and memory copy/kernel processing concurrency. For GPUs that have one or more copy engines, host to/from device memory copy can be performed while the SMs are processing kernels. Within a given stream, operations are performed in sequential order, but operations in different streams may be performed in parallel \citep{CUDAintro}.

To take advantage of CPU/GPU concurrency as depicted in Figure \ref{fig:cpu_gpu}, when performing memory copies as well as kernel launches, asynchronous memory copy must be used. Mapped pinned memory can be used to overlap PCI Express transfers and kernel processing.

\begin{figure}[!htb]
	\centering
	\includegraphics[totalheight=0.3\textwidth]{Figures/cpu_gpu.png}
	\caption{\fontfamily{pag}\selectfont CPU/GPU concurrency \citep{CUDAHand}.}
	\label{fig:cpu_gpu}
\end{figure}

CUDA compute capabilities above 2.0 are capable of concurrently running multiple kernels, provided they are launched in different streams and have block sizes that are small enough so a single kernel will not fill the whole GPU.

By using multiple streams, we broke the kernel computation into chunks and overlap the memory copies with kernel execution. The new improved implementation might have the execution timeline as shown in Figure \ref{fig:streams} in which empty boxes represent time when one stream is waiting to execute an operation that it cannot overlap with the other stream's operation.

\begin{figure}[!htb]
	\centering
	\includegraphics[totalheight=0.32\textheight]{Figures/streams.png}
	\caption{\fontfamily{pag}\selectfont Timeline of intended application execution using two independent streams.}
	\label{fig:streams}
\end{figure}
% \begin{lstlisting}[language=C++, caption={Combined asynchronous memory copy and multiple streams}, captionpos=t]
% // enqueue copies of dbQuad in stream0 and stream1
% cudaMemcpyAsync(cudaPtr.dbQuad0, hostPtr.dbQuad0,
% 		num0 * sizeof(uint),
% 		cudaMemcpyHostToDevice,
% 		stream[0]);
% cudaMemcpyAsync(cudaPtr.dbQuad1, hostPtr.dbQuad1,
% 		num2 * sizeof(uint),
% 		cudaMemcpyHostToDevice,
% 		stream[1]);
% 
% // enqueue copies of hashQuad in stream0 and stream1
% cudaMemcpyAsync(cudaPtr.hashQuad0, hostPtr.hashQuad0, num0 * sizeof(uint), cudaMemcpyHostToDevice, stream[0]);
% cudaMemcpyAsync(cudaPtr.hashQuad1, hostPtr.hashQuad1, num1 * sizeof(uint), cudaMemcpyHostToDevice, stream[1]);
% 
% // enqueue kernels in stream0 and stream1
% msvSIMDKernel0<<<numBlocks, KERNEL_BLOCKSIZE, 0, stream[0]>>>();
% msvSIMDKernel0<<<numBlocks, KERNEL_BLOCKSIZE, 0, stream[1]>>>();
% 
% // enqueue copies of result from device to locked memory
% cudaMemcpyAsync(hostSC0, cudaSC0, num0 * sizeof(float), cudaMemcpyDeviceToHost, stream[0]);
% cudaMemcpyAsync(hostSC1, cudaSC1, num1 * sizeof(float), cudaMemcpyDeviceToHost, stream[1]);
% \end{lstlisting}

Running the improved program using pinned memory, asynchronous memory copy and two streams reveals the time drops from 16.45s to just 9.46s; a quite significant drop of over 73.8\% in execution time.

\subsection{Sorting the database}
\label{dbsort}
As described in Section \ref{MSVsub}, the MSV filter function is sensitive to the length of a target sequence, which determines the execution times of the main \emph{for} loop in Algorithm \ref{MSV-SIMD}. 

The target sequence database could contain many sequences with different lengths. The NCBI NR database used in this thesis consists of over 38 million sequences with sequence lengths varying from 6 to 41,943 amino acids \citep{NCBI}.

This brings a problem for parallel processing of threads on the GPU: one thread could be processing a sequence of several thousands of residues while another might be working on a sequence of just a few. As a result, the thread that finishes first might be idle while the long sequence is being handled. Furthermore, unless care is taken when assigning sequences to threads, this effect might be compounded by the heavily unbalanced workload among threads.

In order to achieve high efficiency for task-based parallelism, the run time of all threads in a thread block should be roughly identical. Therefore the database is converted with sequences being sorted by length. Thus, for two adjacent threads in a thread warp, the difference value between the lengths of the associated sequences is minimized, thereby balancing a similar workload over threads in a warp.

\subsubsection*{Block reading}
Many research implementations were not concerned with practical matters. They simply loaded the whole database data into the memories of the CPU host and the GPU device. A large database, like NCBI NR database, is more than 24GB in size, being too large to load into memory of most machines.

Given that the GPU global memory is much less than the CPU host, we use the size of the GPU global memory as the basis to decide the size of the sequence block while reading the database.

\subsubsection*{Descending order}
The memory pools for database sequences both in the CPU host and the GPU device are dynamically allocated at run time. The pools may be required to be reallocated due to more space needed for the current data block than the last one. If the pool allocated at the first time is the largest one during execution, then the overhead of reallocation will be saved. Hence, the descending order is used for sorting the database.

\subsubsection*{Performance improved}
The CUDA profiling tool nvprof \citep{Profiler} was used to understand and optimize the performance of the MSV GPU application cudaHmmsearch. The nvprof command was used as follows:

\begin{quote}
\fontfamily{phv}\fontseries{m}\selectfont
 \# nvprof  ./cudaHmmsearch globins4.hmm uniprot\_sprot.fasta
\end{quote}

We used the above command twice for profiling cudaHmmsearch execution before and after the target database was sorted. Table \ref{tab.nvprof1} and \ref{tab.nvprof2} list the profiling results respectively. The experiments were done in the environment described in Section \ref{bench}. The globins4.hmm is the Globin4 profile HMM and uniprot\_sprot.fasta is the SP201309 database, which are also described in Section \ref{bench}.

\begin{table}[H]
\centering
\begin{tabular}{|c|c|c|c|c|c|c|}\hline
\shortstack{\textbf{Time(\%)}} & \shortstack{\textbf{Time}} & \shortstack{\textbf{Calls}} & \shortstack{\textbf{Avg}} & \shortstack{\textbf{Min}} & \shortstack{\textbf{Max}} & \shortstack{\textbf{Name}} \\\hline
91.61\% & 4.27108s & 134 & 31.874ms & 5.9307ms & 137.67ms & msvSIMDKernel\\\hline
8.37\% & 390.01ms & 271 & 1.4391ms & 704ns& 23.027ms& [CUDA memcpy HtoD]\\\hline
0.01\% & 556.21us & 134 & 4.1500us & 1.7280us & 5.9840us& [CUDA memcpy DtoH]\\\hline
0.01\% & 491.23us & 134 & 3.6650us & 3.4880us& 4.0640us& [CUDA memset]\\\hline
\end{tabular}
\caption{\fontfamily{pag}\selectfont\textbf{Profiling result of before sorting database.} Each row is the statistics of profiling result for the function named in the `\textbf{Name}'. The statistics includes the percentage of running time, the running time, the number of called times, as well as the average, minimum, and maximum time.\label{tab.nvprof1}}
\end{table}

\begin{table}[H]
\centering
\begin{tabular}{|c|c|c|c|c|c|c|}\hline
\shortstack{\textbf{Time(\%)}} & \shortstack{\textbf{Time}} & \shortstack{\textbf{Calls}} & \shortstack{\textbf{Avg}} & \shortstack{\textbf{Min}} & \shortstack{\textbf{Max}} & \shortstack{\textbf{Name}} \\\hline
97.41\% & 2.07263s & 134 & 15.467ms & 29.056us & 194.91ms & msvSIMDKernel\\\hline
2.54\% & 54.115ms & 271 & 199.69us & 704ns& 23.013ms& [CUDA memcpy HtoD]\\\hline
0.03\% & 550.53us & 134 & 4.1080us & 1.6640us & 4.6720us& [CUDA memcpy DtoH]\\\hline
0.02\% & 474.90us & 134 & 3.5440us & 672ns& 4.0320us& [CUDA memset]\\\hline
\end{tabular}
\caption{\fontfamily{pag}\selectfont\textbf{Profiling result of after sorting database.} The meaning of each column is the same as Table \ref{tab.nvprof1}\label{tab.nvprof2}}
\end{table}

From the result of profiling, we can see the performance has been increased, which is clearly shown in two ways:
\begin{enumerate}
 \item the ratio of the msvSIMDKernel run-time to the total increased from 91.61\% to 97.41\%;
 \item the  msvSIMDKernel run-time decreased from 4.27108s to 2.07263s and the time of the memory copy from Host to Device (CUDA memcpy HtoD) decreased from 390.01ms to 54.115ms.
\end{enumerate}

This approach has the advantage of being both effective and quite straightforward as a large 129\% performance improvement can be gained over the unsorted database without changing the GPU kernel in any way (see Table \ref{tab.opt} and Figure \ref{fig:imp}). For the 24GB NCBI NR database used in these experiments, only 6 minutes were taken for sorting. Further, the sorted database can still be usable for other applications, making the one-time cost of sorting it negligible.

\subsection{Distributing workload}
\label{workload}
After launching the GPU kernel, the CPU must wait for the GPU to finish before copying back the result. This is accomplished by calling \emph{cudaStreamSynchronize}(\emph{stream}). We can get further improvement by distribute some work from the GPU to the CPU while the CPU is waiting. In a protein database, the sequences with the longest or the shortest length are very few. According to Swiss-Prot database statistics \citep{Swiss-Prot}, the percentage of sequences with length $>$ 2500 is only 0.2\%. Considering the length distribution of database sequences and based on the descending sorted database discussed in Section \ref{dbsort}, we assigned the first part of data with longer lengths to the CPU. By this way, we can save both the GPU global memory allocated for sequences and the overheads of memory transfer.

The compute power of the CPU and the GPU should be taken into consideration in order to balance the workload distribution between them. The distribution policy calculates a ratio \emph{R} of the number of database sequences assigned to the CPU, which is calculated as

\begin{equation*}
   R = \frac{f_C}{N_Gf_G + f_C}
\end{equation*}

where $f_G$ and $f_C$ are the core frequencies of the GPU and the CPU respectively, $N_G$ is the number of GPU multiprocessors.

%----------------------------------------------------------------------------------------

\subsection{Miscellaneous consideration}
This sections discusses various small-scale optimization and explains some techniques not suited to the MSV implementation.

\subsection*{Data type for register memory}
\label{register}
In order to reduce the register pressure in the CUDA kernel, we may consider using unsigned 8-bit char type (u8) instead of 32-bit int type (u32). Declaring the registers as u8 results in sections of code to shift and mask data. The extract data macros are deliberately written to mask off the bits that are not used, so this is entirely unnecessary. In fact, around four times the amount of code will be generated if using an u8 type instead of an u32 type.

Changing the u8 definition to an u32 definition benefits from eliminating huge numbers of instructions. It seems potentially to waste some register space. In practice, CUDA implements u8 registers as u32 registers, so this does not actually cost anything extra in terms of register space \citep{cook}.

\subsubsection*{Branch divergence}
\label{branch}

A warp executes one common instruction at a time, so full efficiency is realized when all 32 threads of a warp agree on their execution path. If threads of a warp diverge via a data-dependent conditional branch, the warp serially executes each branch path taken, disabling threads that are not on that path, and when all paths complete, the threads converge back to the same execution path. Branch divergence occurs only within a warp; different warps execute independently regardless of whether they are executing common or disjoint code paths. So the divergence results in some slowdown.

Since the implementation has changed u8 type in register memory to u32 type, the test for the overflow condition is not needed any more. This not only saves several instructions, but also avoids the issue of branch divergence.

\subsubsection*{Constant memory}
\label{constant}
Constant memory is as fast as reading from a register as long as all threads in a warp read the same 4-byte address. Constant memory does not support, or benefit from, coalescing, as this involves threads reading from different addresses. Thus, parameters used by all threads, such as $base$, $t_{jb}$, $t_{ec}$, are stored into constant memory.

\subsubsection*{Shared memory}
\label{shared}
In terms of speed, shared memory is perhaps 10x slower than register accesses but 10x faster than accesses to global memory. However, some disadvantages apply to shared memory.

\begin{itemize}
 \item Unlike the L1 cache, the shared memory has a per-block visibility, which would mean having to duplicate the data for every resident block on the SM.
 \item Data must be loaded from global to shared memory in GPU kernel and can not be uploaded to shared memory directly from the host memory.
 \item Shared memory is well suited to exchange data between CUDA threads within a block. As described in subsection \ref{impl}, task-based parallelism is applied without the need for inter-thread communications, which also saves the cost of synchronization \emph{\_\_syncthreads}() among threads.
\end{itemize}

Because of these disadvantages, the MSV implementation does not use shared memory.

\subsubsection*{Kernel launch configuration}
\label{launch}
Since the MSV implementation does not use shared memory (as explained above), the following dynamic kernel launch configuration is used to prefer the larger L1 cache and smaller shared memory so as to further improve memory throughput.
\begin{quote}
\fontfamily{phv}\fontseries{m}\selectfont
 cudaFuncSetCacheConfig(msvSIMDKernel, cudaFuncCachePreferL1);
\end{quote}

%----------------------------------------------------------------------------------------

\section{Conclusion of optimization}

This section briefly reviews the all the optimization approaches discussed in this chapter thus far, and summarizes the steps to gain better performance for CUDA programming.

\subsection*{Six steps to better performance}
\begin{enumerate}
 \item Assessing the application\\
 In order to benefit from any modern processor architecture, including GPUs, the first step is to assess the application to identify the hotspots [MSV filter in Section \ref{hmmsearch}], and to identify which type of parallelism [Task-based parallelism in Section \ref{impl}] is better suited to the application.
 
 \item Profiling the application\\
 NVIDIA provides profiling tools to help identify hotspots and compile a list of candidates for parallelization or optimization on CUDA-enabled GPUs [nvprof in Section \ref{dbsort}], as detailed in Section \ref{cudaTools}. 
 Intel provides VTune Amplifier XE to collect a rich set of data to tune CPU and GPU compute performance at \url{https://software.intel.com/en-us/intel-vtune-amplifier-xe}.
 
 \item Optimizing memory usage\\
 Optimizing memory usage starts with minimizing data transfers both in size [Data-base sorted in Section \ref{dbsort}, workload distribution in Section \ref{workload}] and time [Pinned Memory in Section \ref{pin}] between the host and the device [Asynchronous memory copy in Section \ref{asyn}]. Be careful with the CUDA memory hierarchy: register memory [Section \ref{register}], local memory, shared memory [Section \ref{shared}], global memory, constant memory [Section \ref{constant}] and texture memory [Section \ref{tex}], and combine these memories to best suit the application [Kernel launch configuration in Section \ref{launch}]. Sometimes, the best optimization might even be to avoid any data transfer in the first place by simply recomputing the data whenever it is needed.\\
 The next step in optimizing memory usage is to organize memory accesses according to the optimal memory access patterns. This optimization is especially important for coalescing global memory accesses [Section \ref{global}].
 
 \item Optimizing instruction usage\\
 This principle suggests using SIMD Video Instructions [Section \ref{video}] and trading precision for speed when it does not affect the end result, such as using intrinsic instead of regular functions or single precision instead of double precision [HMMER3 in Section \ref{SSE2}]. Particular attention should be paid to the control flow instructions [Branch divergence in Section \ref{branch}].
 
 \item Maximizing parallel execution\\ 
 The application should maximize parallel execution at a higher level by explicitly exposing concurrent execution on the device through streams [Section \ref{asyn}], as well as maximizing concurrent execution between the CPU host  [Database sorted in Section \ref{dbsort}] and the GPU device [Workload distribution in Section \ref{workload} and SIMD Video Instructions in Section \ref{video}].
 
 \item Considering the existing libraries\\ 
 Many existing GPU-optimized libraries \citep{CUDAlibs} such as cuBLAS \citep{cuBLAS}, MAGMA \citep{MAGMA}, ArrayFire \citep{ArrayFire}, or Thrust \citep{thrust}, are available to make the expression of parallel code as simple as possible.
 
\end{enumerate}


